% wedoc.tex V1.0, 17 June 2010

\documentclass[times]{weauth}

\usepackage{moreverb}

%\usepackage[dvips,colorlinks,bookmarksopen,bookmarksnumbered,citecolor=red,urlcolor=red]{hyperref}

\newcommand\BibTeX{{\rmfamily B\kern-.05em \textsc{i\kern-.025em b}\kern-.08em
T\kern-.1667em\lower.7ex\hbox{E}\kern-.125emX}}

\def\volumeyear{2010}

\begin{document}

\runningheads{A.~N.~Other}{A demonstration of the \journalabb\
class file}

\articletype{RESEARCH ARTICLE}

\title{A demonstration of the \LaTeXe\ class file for
\itshape{\journalnamelc}\footnotemark[2]}

\author{A.~N.~Other}

\address{John Wiley \& Sons, Ltd, The Atrium, Southern Gate, Chichester,
West Sussex, PO19~8SQ, UK}

\corraddr{Journals Production Department, John Wiley \& Sons, Ltd,
The Atrium, Southern Gate, Chichester, West Sussex, PO19~8SQ, UK.}

\begin{abstract}
This paper describes the use of the \LaTeXe\
\textsf{\journalclass} class file for setting papers for
\emph{\journalnamelc}.
\end{abstract}

\keywords{class file; \LaTeXe; \emph{\journalabb}}

\maketitle

\footnotetext[2]{Please ensure that you use the most up to date
class file,
available from the WE Home Page at\\
%\href{http://www3.interscience.wiley.com/journal/6276/home}{\texttt{http://www3.interscience.wiley.com/journal/6276/home}}
}

\section{Introduction}
Many authors submitting to research journals use \LaTeXe\ to
prepare their papers. This paper describes the
\textsf{\journalclass} class file which can be used to convert
articles produced with other \LaTeXe\ class files into the correct
form for publication in \emph{\journalnamelc}.

The \textsf{\journalclass} class file preserves much of the
standard \LaTeXe\ interface so that any document which was
produced using the standard \LaTeXe\ \textsf{article} style can
easily be converted to work with the \textsf{\journalclassshort}
style. However, the width of text and typesize will vary from that
of \textsf{article.cls}; therefore, \emph{line breaks will change}
and it is likely that displayed mathematics and tabular material
will need re-setting.

In the following sections we describe how to lay out your code to
use \textsf{\journalclass} to reproduce the typographical look of
\emph{\journalnamelc}. However, this paper is not a guide to
using \LaTeXe\ and we would refer you to any of the many books
available (see, for example, \cite{R1,R2,R3}).

\section{The Three Golden Rules}

Before we proceed, we would like to stress \emph{three golden
rules} that need to be followed to enable the most efficient use
of your code at the typesetting stage:
\begin{enumerate}
\item[(i)] keep your own macros to an absolute minimum;

\item[(ii)] as \TeX\ is designed to make sensible spacing
decisions by itself, do \emph{not} use explicit horizontal or
vertical spacing commands, except in a few accepted (mostly
mathematical) situations, such as \verb"\," before a
differential~d, or \verb"\quad" to separate an equation from its
qualifier;

\item[(iii)] follow the \emph{\journalnamelc} reference style.
\end{enumerate}

\section{Getting Started} The \textsf{\journalclassshort} class file should run
on any standard \LaTeXe\ installation. If any of the fonts, style
files or packages it requires are missing from your installation,
they can be found on the \emph{\TeX\ Collection} DVDs or from
CTAN.

\emph{\journalnamelc} is published using Times fonts and this is
achieved by using the \verb"times"
option as\\
\verb"\documentclass[times]{weauth}".

\noindent If for any reason you have a problem using Times you can
easily resort to Computer Modern fonts by removing the
\verb"times" option.

\begin{figure}
\setlength{\fboxsep}{0pt}%
\setlength{\fboxrule}{0pt}%
\begin{center}
\begin{boxedverbatim}
\documentclass[times]{weauth}
%\documentclass[times,doublespace]{weauth}%For paper submission

\begin{document}

\runningheads{<Initials and Surnames>}{<Short title>}

\articletype{<Article type>}

\title{<Initial cap, lower case>}

\author{<An Author\affil{1},
Someone Else\affil{2}\ and Perhaps Another\affil{1}>}

\address{<\affilnum{1}First author's address
(in this example it is the same as the third author)
\affilnum{2}Second author's address>}

\corraddr{<Corresponding author's address>.
E-mail: <corresponding author's email address>}

%\cgs{<Contract/grant sponsor name (no number)>}
%\cgsn{<Contract/grant sponsor name>}{<number>}

\begin{abstract}
<Text>
\end{abstract}

\keywords{<List keywords>}

\maketitle

\section{Introduction}
.
.
.
\end{boxedverbatim}
\end{center}
\caption{Example header text.\label{F1}}
\end{figure}


\section{The Article Header Information}
The heading for any file using \textsf{\journalclass} is shown in
Figure~\ref{F1}.

\subsection{Remarks}
\begin{enumerate}
\item[(i)] In \verb"\runningheads" use `\emph{et~al.}' if there
are three or more authors.

\item[(ii)] Note the use of \verb"\affil" and \verb"\affilnum" to
link names and addresses. The details for the corresponding
author should be included in the \verb"\corraddr" command.

\item[(iii)] For submitting a double-spaced manuscript, add
\verb"doublespace" as an option to the documentclass line.

\item[(iv)] Use \verb"\cgs" for giving details of financial
sponsors; alternatively use \verb"\cgsn" if the grant number is
also to be included.
%These details will be printed as a footnote,
%with `Contract/grant sponsor:' and `contract/grant number:'
%inserted in the appropriate places.

\item[(v)] The abstract should be capable of standing by itself,
in the absence of the body of the article and of the bibliography.
Therefore, it must not contain any reference citations.

\item[(vi)] Keywords are separated by semicolons.
\end{enumerate}

\begin{figure}
\setlength{\fboxsep}{0pt}%
\setlength{\fboxrule}{0pt}%
\begin{center}
\begin{boxedverbatim}
\begin{table}
\caption{<Table caption>}
\centering
\small
\begin{tabular}{<table alignment>}
\toprule
<column headings>\\
\midrule
<table entries
(separated by & as usual)>\\
<table entries>\\
.
.
.\\
\bottomrule
\end{tabular}
\end{table}
\end{boxedverbatim}
\end{center}
\caption{Example table layout.\label{F2}}
\end{figure}

\section{The Body of the Article}

\subsection{Mathematics} \textsf{\journalclass} makes the full
functionality of \AmS\/\TeX\ available. We encourage the use of
the \verb"align", \verb"gather" and \verb"multline" environments
for displayed mathematics. \textsf{amsthm} is used for setting
theorem-like and proof environments. The usual \verb"\newtheorem"
command needs to be used to set up the environments for your
particular document.

\subsection{Figures and Tables} \textsf{\journalclass} includes the
\textsf{graphicx} package for handling figures.

Figures are called in as follows:
\begin{verbatim}
\begin{figure}
\centering
\includegraphics{<figure name>}
\caption{<Figure caption>}
\end{figure}
\end{verbatim}

For further details on how to size figures, etc., with the
\textsf{graphicx} package see, for example, \cite{R1}
or \cite{R3}. If figures are available in an
acceptable format (for example, .eps, .ps) they will be used but a
printed version should always be provided. \medbreak

The standard coding for a table is shown in Figure~\ref{F2}.

\subsection{Cross-referencing}
The use of the \LaTeX\ cross-reference system
for figures, tables, equations, etc., is encouraged
(using \verb"\ref{<name>}" and \verb"\label{<name>}").

\subsection{Acknowledgements} An Acknowledgements section is started with \verb"\ack" or
\verb"\acks" for \textbf{ACKNOWLEDGEMENT} or
\textbf{ACKNOWLEDGEMENTS}, respectively. It must be placed just
before the References.

\subsection{Bibliography}
The normal commands for producing the reference list are:
\begin{verbatim}
\begin{thebibliography}{99}
\bibitem{<x-ref label>}
         <Reference details>
.
.
.
\end{thebibliography}
\end{verbatim}
where \verb"\bibitem{x-ref label}"
corresponds to \verb"\cite{x-ref label}" in the body of the article
and \verb"{99}" is the widest such number expected and determines
the width of the number column in the reference list.

Please note that the file \textsf{wileyj.bst} is available from
the same download page for those authors using \BibTeX.

\subsection{Double Spacing}
If you need to double space your document for submission please
use the \verb+doublespace+ option as shown in the sample layout in
Figure~\ref{F1}.

\section{Support for \textsf{\journalclass}}
We offer on-line support to participating authors. Please contact
us via e-mail at\\
%\href{mailto:weauth-cls@wiley.co.uk}{\texttt{weauth-cls@wiley.co.uk}}.

We would welcome any feedback, positive or otherwise, on your
experiences of using \textsf{\journalclass}.

\section{Copyright Statement}
Please  be  aware that the use of  this \LaTeXe\ class file is
governed by the following conditions.

\subsection{Copyright}
Copyright \copyright\ \volumeyear\ John Wiley \& Sons, Ltd, The
Atrium, Southern Gate, Chichester, West Sussex, PO19~8SQ, UK. All
rights reserved.

\subsection{Rules of Use}
This class file is made available for use by authors who wish to
prepare an article for publication in \emph{\journalnamelc}
published by John Wiley \& Sons, Ltd. The user may not exploit any
part of the class file commercially.

This class file is provided on an \emph{as is}  basis, without
warranties of any kind, either express or implied, including but
not limited to warranties of title, or implied  warranties of
merchantablility or fitness for a particular purpose. There will
be no duty on the author[s] of the software or  John Wiley \&
Sons, Ltd to correct any errors or defects in the software. Any
statutory  rights you may have remain unaffected by your
acceptance of these rules of use.

\ack This class file was developed by Sunrise Setting Ltd,
Torquay, Devon, UK. Website:\\
%\href{http://www.sunrise-setting.co.uk}{\texttt{www.sunrise-setting.co.uk}}

\begin{thebibliography}{9}

\bibitem{R1} Kopka~H, Daly~PW. 2003. \emph{A Guide to \LaTeX} (4th~edn).
Addison-Wesley.

\bibitem{R2} Lamport~L. 1994. \emph{\LaTeX: a Document Preparation System} (2nd~edn).
Addison-Wesley.

\bibitem{R3} Mittelbach~F, Goossens~M. 2004. \emph{The \LaTeX\ Companion}
(2nd~edn). Addison-Wesley.
\end{thebibliography}
\end{document}
